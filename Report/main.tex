\documentclass{article}
\usepackage{graphicx}

\title{ \textbf{Exploration of different Optimization Paradigms for the Sports Tournament Scheduling (STS) problem}}
\author{Matteo Preda - matteo.preda2@studio.unibo.it \\ Raffaele Sali - raffaele.sali@studio.unibo.it \\ Marco Zocco Ramazzo - marco.zoccoramazzo@studio.unibo.it}
\date{September 2025}

\begin{document}

\maketitle

\section{Introduction}
The report provides an in-depth study of the Sports Tournament Scheduling (STS) problem, emphasizing the formulation and comparison of different optimization models. The models examined in this project are built upon a shared formalization, which is outlined in the following section. The main goal of the presented work is to build a unified infrastructure for modeling, analyzing, and solving the STS problem, by considering the different techniques in these areas.  

\subsection{Model Formalization}

\paragraph{Input Parameters.}
The STS problem is defined by the following parameters, which are common to all models:
\begin{itemize}
    \item $n$: Number of teams (even integer)
    \item $w = n-1$: Number of weeks
    \item $p = n/2$: Number of periods (matches per week)
    \item Teams are indexed by $t \in \{1, 2, \ldots, n\}$
    \item Weeks are indexed by $w \in \{1, 2, \ldots, n-1\}$
    \item Periods are indexed by $p \in \{1, 2, \ldots, n/2\}$
\end{itemize}

\paragraph{Objective Variable and its Bounds.}
The objective function, which is only related to the optimization version, represents the maximum imbalance in home and away games for any team, with the goal of minimizing it:
\[
M = \max_{t \in \{1, \ldots, n\}} |h_t - a_t|
\]
where $h_t$ and $a_t$ represent the number of home and away games played by the team $t$, respectively. 
The theoretical upper bound for $M$ is $n-1$, while the lower bound for $M$ is $1$, since having an even $n$ implies that the matches each team plays (equal to $w$) are odd. For this reason, $h_t$ and $a_t$ must be different.

\paragraph{Constraints.}
All models rest on the following core constraints:
\begin{enumerate}
    \item Every team plays every other team only once in the tournament.
    \item Every team plays once a week.
    \item Each period in each week has exactly one game.
    \item Every team plays at most twice in the same period over the tournament.
    \item A team cannot play against itself.
\end{enumerate}

\paragraph{Pre-processing steps and Symmetry Breaking constraints.}
To increase solver performance, preprocessing steps might have been included in the examined models:
\begin{itemize}
    \item Building the model using a solver-independent language, like AMPL, which may require additional preprocessing time before the actual solving phase.
    \item Introducing implied constraints, such as the number of matches per team.
    \item Fixing the date of some matches to break team symmetries.
\end{itemize}

\subsection{Code Availability}
The project work can be found at \url{https://github.com/mpreda01/CDMO_project}.


\section{CP Model}

\subsection{Decision variables}

\subsection{Objective function}

\subsection{Constraints}

\subsection{Validation}

\section{SAT Model}

\subsection{Decision variables}

\subsection{Objective function}

\subsection{Constraints}

\subsection{Validation}

\section{SMT Model}

\subsection{Decision variables}

\subsection{Objective function}

\subsection{Constraints}

\subsection{Validation}

\section{MIP Model}

\subsection{Decision variables}

\subsection{Objective function}

\subsection{Constraints}

\subsection{Validation}

\section{Conclusions}

\section{Authenticity and Author Contribution Statement}

\section{References}

\end{document}
